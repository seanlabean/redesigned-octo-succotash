% set \longtrue to make a more detailed CV.
% set \longfalse to make a shorter CV
\newif\iflong
\longtrue

\newif\iflongexperience

\iflong
	\longexperiencetrue
\fi
\longtrue

%%%%%%%%%%%%%%%%%%%%%%%%%%%%%%%%%%%%%%%%%%%%%%%%%%%%%%%%%%%%%%%%%%%%%%%%
%%%%%%%%%%%%%%%%%%%%%% Simple LaTeX CV Template %%%%%%%%%%%%%%%%%%%%%%%%
%%%%%%%%%%%%%%%%%%%%%%%%%%%%%%%%%%%%%%%%%%%%%%%%%%%%%%%%%%%%%%%%%%%%%%%%
\documentclass[10pt]{article}

% for Roman serrif font
%\usepackage{times}
%\renewcommand{\familydefault}{\sfdefault}
% This is a helpful package that puts math inside length specifications
\usepackage{calc}
\usepackage{comment}

% Simpler bibsection for CV sections
% (thanks to natbib for inspiration)
\makeatletter
\newlength{\bibhang}
\setlength{\bibhang}{1em} %1em}
\newlength{\bibsep}
 {\@listi \global\bibsep\itemsep \global\advance\bibsep by\parsep}
\newenvironment{bibsection}%
        {\begin{enumerate}{}{%
%        {\begin{list}{}{%
       \setlength{\leftmargin}{\bibhang}%
       \setlength{\itemindent}{-\leftmargin}%
       \setlength{\itemsep}{\bibsep}%
       \setlength{\parsep}{\z@}%
        \setlength{\partopsep}{0pt}%
        \setlength{\topsep}{0pt}}}
        {\end{enumerate}\vspace{-.6\baselineskip}}
%        {\end{list}\vspace{-.6\baselineskip}}
\makeatother

% Layout: Puts the section titles on left side of page
\reversemarginpar




%
%         PAPER SIZE, PAGE NUMBER, AND DOCUMENT LAYOUT NOTES:
%
% The next \usepackage line changes the layout for CV style section
% headings as marginal notes. It also sets up the paper size as either
% letter or A4. By default, letter was used. If A4 paper is desired,
% comment out the letterpaper lines and uncomment the a4paper lines.
%
% As you can see, the margin widths and section title widths can be
% easily adjusted.
%
% ALSO: Notice that the includefoot option can be commented OUT in order
% to put the PAGE NUMBER *IN* the bottom margin. This will make the
% effective text area larger.
%
% IF YOU WISH TO REMOVE THE ``of LASTPAGE'' next to each page number,
% see the note about the +LP and -LP lines below. Comment out the +LP
% and uncomment the -LP.
%
% IF YOU WISH TO REMOVE PAGE NUMBERS, be sure that the includefoot line
% is uncommented and ALSO uncomment the \pagestyle{empty} a few lines
% below.
%

%% Use these lines for letter-sized paper
\usepackage[paper=letterpaper,
            %includefoot, % Uncomment to put page number above margin
            marginparwidth=1.2in,     % Length of section titles
            marginparsep=.05in,       % Space between titles and text
            margin=1in,               % 1 inch margins
            includemp]{geometry}

%% Use these lines for A4-sized paper
%\usepackage[paper=a4paper,
%            %includefoot, % Uncomment to put page number above margin
%            marginparwidth=30.5mm,    % Length of section titles
%            marginparsep=1.5mm,       % Space between titles and text
%            margin=25mm,              % 25mm margins
%            includemp]{geometry}

%% More layout: Get rid of indenting throughout entire document
\setlength{\parindent}{0in}

\usepackage[shortlabels]{enumitem}

%% Reference the last page in the page number
%
% NOTE: comment the +LP line and uncomment the -LP line to have page
%       numbers without the ``of ##'' last page reference)
%
% NOTE: uncomment the \pagestyle{empty} line to get rid of all page
%       numbers (make sure includefoot is commented out above)
%
\usepackage{fancyhdr,lastpage}
\pagestyle{fancy}
%\pagestyle{empty}      % Uncomment this to get rid of page numbers
\fancyhf{}\renewcommand{\headrulewidth}{0pt}
\fancyfootoffset{\marginparsep+\marginparwidth}
\newlength{\footpageshift}
\setlength{\footpageshift}
          {0.5\textwidth+0.5\marginparsep+0.5\marginparwidth-2in}
\lfoot{\hspace{\footpageshift}%
       \parbox{4in}{\, \hfill %
                    \arabic{page} of \protect\pageref*{LastPage} % +LP
%                    \arabic{page}                               % -LP
                    \hfill \,}}

% Finally, give us PDF bookmarks
\usepackage{color,hyperref}
\definecolor{darkblue}{rgb}{0.0,0.0,0.9}
\hypersetup{colorlinks,breaklinks,
            linkcolor=darkblue,urlcolor=darkblue,
            anchorcolor=darkblue,citecolor=darkblue}

%%%%%%%%%%%%%%%%%%%%%%%% End Document Setup %%%%%%%%%%%%%%%%%%%%%%%%%%%%
% =================================================================================================%

%%%%%%%%%%%%%%%%%%%%%%%%%%% Helper Commands %%%%%%%%%%%%%%%%%%%%%%%%%%%%

% The title (name) with a horizontal rule under it
% (optional argument typesets an object right-justified across from name
%  as well)
%
% Usage: \makeheading{name}
%        OR
%        \makeheading[right_object]{name}
%
% Place at top of document. It should be the first thing.
% If ``right_object'' is provided in the square-braced optional
% argument, it will be right justified on the same line as ``name'' at
% the top of the CV. For example:
%
%       \makeheading[\emph{Curriculum vitae}]{Your Name}
%
% will put an emphasized ``Curriculum vitae'' at the top of the document
% as a title. Likewise, a picture could be included:
%
%   \makeheading[\includegraphics[height=1.5in]{my_picutre}]{Your Name}
%
% the picture will be flush right across from the name.
\newcommand{\makeheading}[2][]%
        {\hspace*{-\marginparsep minus \marginparwidth}%
         \begin{minipage}[t]{\textwidth+\marginparwidth+\marginparsep}%
             {\large \bfseries #2 \hfill #1}\\[-0.15\baselineskip]%
                 \rule{\columnwidth}{1pt}%
         \end{minipage}}

% The section headings
%
% Usage: \section{section name}
\renewcommand{\section}[1]{\pagebreak[3]%
    \hyphenpenalty=10000%
    \vspace{1.3\baselineskip}%
    \phantomsection\addcontentsline{toc}{section}{#1}%
    \noindent\llap{\scshape\smash{\parbox[t]{\marginparwidth}{\raggedright #1}}}%
    \vspace{-\baselineskip}\par}

% An itemize-style list with lots of space between items
\newenvironment{outerlist}[1][\enskip\textbullet]%
        {\begin{itemize}[#1,leftmargin=*]}{\end{itemize}%
         \vspace{-.6\baselineskip}}

% An environment IDENTICAL to outerlist that has better pre-list spacing
% when used as the first thing in a \section
\newenvironment{lonelist}[1][\enskip\textbullet]%
        {\begin{list}{#1}{%
        \setlength{\partopsep}{0pt}%
        \setlength{\topsep}{0pt}}}
        {\end{list}\vspace{-.6\baselineskip}}

% An itemize-style list with little space between items
\newenvironment{innerlist}[1][\enskip\textbullet]%
        {\begin{itemize}[#1,leftmargin=*,parsep=0pt,itemsep=0pt,topsep=0pt,partopsep=0pt]}
        {\end{itemize}}

% An environment IDENTICAL to innerlist that has better pre-list spacing
% when used as the first thing in a \section
\newenvironment{loneinnerlist}[1][\enskip\textbullet]%
        {\begin{itemize}[#1,leftmargin=*,parsep=0pt,itemsep=0pt,topsep=0pt,partopsep=0pt]}
        {\end{itemize}\vspace{-.6\baselineskip}}

% To add some paragraph space between lines.
% This also tells LaTeX to preferably break a page on one of these gaps
% if there is a needed pagebreak nearby.
\newcommand{\blankline}{\quad\pagebreak[3]}
\newcommand{\halfblankline}{\quad\vspace{-0.5\baselineskip}\pagebreak[3]}

% Uses hyperref to link DOI
\newcommand\doilink[1]{\href{http://dx.doi.org/#1}{#1}}
\newcommand\doi[1]{doi:\doilink{#1}}

% For \url{SOME_URL}, links SOME_URL to the url SOME_URL
\providecommand*\url[1]{\href{#1}{#1}}
% Same as above, but pretty-prints SOME_URL in teletype fixed-width font
\renewcommand*\url[1]{\href{#1}{\texttt{#1}}}

% For \email{ADDRESS}, links ADDRESS to the url mailto:ADDRESS
\providecommand*\email[1]{\href{mailto:#1}{#1}}
% Same as above, but pretty-prints ADDRESS in teletype fixed-width font
%\renewcommand*\email[1]{\href{mailto:#1}{\texttt{#1}}}

%\providecommand\BibTeX{{\rm B\kern-.05em{\sc i\kern-.025em b}\kern-.08em
%    T\kern-.1667em\lower.7ex\hbox{E}\kern-.125emX}}
%\providecommand\BibTeX{{\rm B\kern-.05em{\sc i\kern-.025em b}\kern-.08em
%    \TeX}}
\providecommand\BibTeX{{B\kern-.05em{\sc i\kern-.025em b}\kern-.08em
    \TeX}}
\providecommand\Matlab{\textsc{Matlab}}




%%%%%%%%%%%%%%%%%%%%%%%% End Helper Commands %%%%%%%%%%%%%%%%%%%%%%%%%%%

%%%%%%%%%%%%%%%%%%%%%%%%% Begin CV Document %%%%%%%%%%%%%%%%%%%%%%%%%%%%

\begin{document}
\makeheading{Sean C. Lewis \hfill \textnormal{Drexel University}}%\hfill Computational Physicist} \

\section{Contact Information}

% NOTE: Mind where the & separators and \\ breaks are in the following
%       table.
%
% ALSO: \rcollength is the width of the right column of the table
%       (adjust it to your liking; default is 1.85in).
%
\newlength{\rcollength}\setlength{\rcollength}{1.4in}%
%
\begin{tabular}[t]{@{}p{\textwidth-\rcollength}p{\rcollength}}
\email{scl63@drexel.edu} \hfill 408-470-0668 \\
Drexel University \hfill \\%3141 Chestnut Street \\
Physics Department \hfill %Philadelphia, PA 19104
\end{tabular}

%\section{Objective}

%Insert text here if you want to
%\begin{innerlist}
%\item More information and auxiliary documents can be found at\\\url{http://www.tedpavlic.com/facjobsearch/}
%\end{innerlist}

\section{Research Interests}
I study the effects of massive O-type star formation. Specifically, I am interested in how O-star feedback mechanisms of radiation, stellar winds, and supernovae expel surrounding natal gas and disrupt subsequent star formation. I utilize the software suite {\bf Torch}--which couples the MHD code {\bf FLASH} with collisional N-body and stellar evolution codes--to design a controlled set of simulations where I control the timing of embedded massive star formation. I am also interested in the puzzle of multiple stellar populations in stellar clusters and how supernovae pollution may contribute to the observed phenomenon.
%My research currently focuses on the physical interaction between supernova ejecta and protoplanetary disks. I use the grid-based magnetohydrodynamics code {\bf FLASH} to model and study these interactions. Specifically, I will examine the stripping of disk mass through ram pressure and flow instabilities, the enrichment of the disk through capture of supernova ejecta, and the potential for accelerated planet formation through disk structure instabilities due to nearby supernovae. In the future, I will apply my knowledge of hydrodynamic and N-body simulations to design and analyze star birthing regions. I hope to continue honing my skills as a computational physicist use my current skill set to great effect at the Los Alamos National Laboratory.

%--------Josh's research. For example use only.--------
%My research focuses on the formation and early evolution of star clusters. I have coupled the MHD code {\bf FLASH} with collisional N-body and stellar evolution codes to study how stellar formation is effected by local stellar feedback. Specifically I study how winds, radiation and supernova quench star formation, eject gas and disrupt clusters. I am also investigating using Poisson sampling of the initial mass function to determine whether star formation is a fundamentally stochastic process without recourse to an upper limit set by the final cluster mass. Finally, I am also interested in the puzzle of multiple stellar populations in globular clusters, including how pressure confinement may contribute to the formation of later generations of stars in starburst galaxy environments.

\section{Education}

\href{}{\textbf{Drexel University}}, Philadelphia, PA
\begin{outerlist}

\item[] ``The Effects of Early Massive Star Formation and A Novel Method for Code Base Transfer Between Voronoi Mesh and Cartesian Adaptive Grid Codes'', \\
		\href{}
			{Ph.D. Physics},
			\emph{Expected Degree:} 2023
			
\item[] M.S.
		\href{}
			{Physics},
			2019
\end{outerlist}

\vspace{.1in}
\textbf{California Polytechnic State University},
San Luis Obispo, CA
\begin{outerlist}

%        \begin{innerlist}
%        \item Thesis Topic: \emph{asdf}
%        \item Advisors:
%              \href{}
%                   {name} and
%              \href{http://www.biostat.umn.edu/~sudiptob/}
%                   {Sudipto Banerjee, Ph.D}
%        \end{innerlist}

\item[] B.S.
        %\href{http://www.apsu.edu/}
             Physics,
             2016
        \begin{innerlist}
		\item Cum Laude    
%Advisor : \href{http://www.apsu.edu/physics/facstaff}{Alex King, Ph.D.}

        \end{innerlist}
\end{outerlist}

% The 'iflong' 'fi' statements in the following are used to disable/enable
% descriptions of research experience when generating CV, depending on 
% need (see if statements at top of file)

\section{Research Experience}

%\textbf{Research Assistant} \hfill {July 2015 - Present}

%\begin{innerlist}

%\item[] Department of Physics, Drexel University - Philadelphia, PA\\
%    Supervisors: Stephen McMillan and Mordecai-Mark Mac Low
%		\begin{innerlist}
%		\item Research utilizes large scale computational simulations to study
%		galaxy formation and evolution on all scales, with a focus on dwarf galaxies.
%		\item Currently implementing a chemodynamics method into the AMR, hydro code 
%		\textsc{Enzo} using star-by-star modeling and including the effects of supernovae,
%		stellar winds, cosmic rays, and full radiative transfer.
%		\item Utilize following code projects: \href{http://enzo-project.org/}{Enzo}, 
%		\href{http://yt-project.org/}{\textit{yt}}, \href{http://flash.uchicago.edu/site/}{FLASH}
%		\end{innerlist}
%\end{innerlist}

%\iflong
\textbf{Graduate Research} \hfill {2018 - Present}
\begin{innerlist}

\item[] Department of Physics, Drexel University\\
		Supervisor: Stephen McMillan, Department Chair
		\iflong
		\begin{innerlist}
		\item \textbf{Thesis:} I design a controlled set of magnetohydrodynamical N-body simulations to explore the effects of O-type star formation on natal gas and star cluster formation.
		\item  Designed a simulation to analyze the survivability of a globular cluster's close encounter with a supermassive binary black hole using the AMUSE N-body dynamics software suite. Presented as a research poster at the 233rd American Astronomical Society meeting in Seattle.
		%\item Designed a simulation of a gas-phase, isothermal protoplanetary disk and its interaction with a supernova blast wave using the hydrodynamics code FLASH.
		\item[]
		\end{innerlist}
		\fi
		
\end{innerlist} 

\textbf{Undergraduate Research Assistant} \hfill {2015 - 2016}
\begin{innerlist}

\item[] Department of Physics, California Polytechnic State Univ. - San Luis Obispo, CA\\
		Supervisor: Vardha Bennert, Ph.D
		\iflong
		\begin{innerlist}
		\item Analyzed active galactic nuclei emission lines with line fitting techniques and   examined the relationship between the doppler broadening of narrow line region emissions and stellar dynamics within the galactic core.
		\end{innerlist}
		\fi
\end{innerlist}
%\fi

%\section{Talks}
%\vspace{-.125in}
%\begin{bibsection}
%
%	\item \textbf{J. Wall}, S. McMillan, M.-M. Mac Low, Ralf Klessen, S. Portegies Zwart, ``Formation and Early Evolution of Star Clusters with Feedback.'' MODEST 2018
%	Santorini Greece (June 2018)

%    \item \textbf{J. Wall}, S. McMillan, M.-M. Mac Low, Ralf Klessen, S. Portegies Zwart, ``Simulating the Formation of Star Clusters.'' MODEST NYC
%    New York, NY USA (July 2016)
    
%    \item \textbf{J. Wall}, S. McMillan, M.-M. Mac Low, S. Portegies Zwart, ``Coupling Hydrodynamics and N-body Codes Using a Gravity Bridge.'' RAMBODY Workshop,
%    Leiden, Netherlands (October 2015)

%\end{bibsection}

%\section{Publications}
%\vspace{-.125in}
%\begin{bibsection}
%	\item  \textbf{J. Wall}, S. McMillan, M.-M. Mac Low, Ralf Klessen, S. Portegies Zwart, ``Collisional N-Body Dynamics Coupled to Self-Gravitating
%	Magnetohydrodynamics Reveals Dynamical Binary Formation.'' ApJ. Submitted. ArXiV link.

%\end{bibsection}

%\section{Conference Publications}
%\vspace{-.125in}
%\begin{bibsection}

%	\item \textbf{J. Wall}, S. McMillan, M.-M. Mac Low, Ralf Klessen, S. Portegies Zwart, ``Formation and Early Evolution of Star Clusters with Feedback.''
%	Star Formation in Space and Time
%	Florence, Italy (June 2017)
	
%  	\item \textbf{J. Wall}, S. McMillan, M.-M. Mac Low, J.C. Iba\~{n}ez-Mej\'{i}a, S. Portegies Zwart, ``Formation of Star Clusters.'' Kobe, Japan, MODEST 2015-S
%    (December 2015)

% 	\item \textbf{J. Wall}, S. McMillan, M.-M. Mac Low, S. Portegies Zwart, ``Coupling Magnetohydrodynamics and N-body Codes Using a Gravity Bridge.''
%    IAU 2015, Honolulu, HI. (August 2015)
  
%  	\item \textbf{J. Wall}, S. McMillan, M.-M. Mac Low, K. Olson, S. Portegies Zwart ``Coupling Magnetohydrodynamics and N-body Codes Using a Gravity Bridge.'' 
%  	MODEST 2015, Concepi\'{o}n, Chile. (March 2015)
  	
%\end{bibsection}

\section{Software Experience}
\vspace{-.125in}
\begin{bibsection}
	\item Astrophysical Multipurpose Software Environment ({\bf AMUSE}). \\
	Publicly available at \url{https://github.com/amusecode/amuse}.
	\item {\bf FLASH} grid-based MHD. \\
	More information at \url{http://flash.uchicago.edu/site/}
	\item {\bf yt} data visualization Python package. \\ 
	More information at \url{http://yt-project.org/}
	%\item Implemented gravity bridge for {\bf FLASH} and any N-body code in {\bf AMUSE}.
	%\item Implemeted {\bf FLASH} stellar wind module.
	%\item Extended {\bf FLASH} {\bf FERVENT} ray-tracing radiation module to include FUV.
	%\item Implemented {\bf FLASH} supernova module.
	%\item Implemented a python, MPI parallel binary analysis package for {\bf AMUSE}.
\end{bibsection}

\section{Teaching Experience}
\begin{innerlist}
\item T.A. Physics 121: Physical Science for Design I \hfill {Fall 2017}
\item T.A. Physics 101: Physics for Engineers I \hfill {Winter 2018}
\item T.A. Physics 102: Physics for Engineers II \hfill {Spring, Summer, Fall 2018}
\item T.A. Physics 201: Physics for Engineers III \hfill {Winter, Spring, Fall 2019}
\end{innerlist}

%\halfblankline

% Add a little space to nudge next ``Conference Publications'' marginpar
% down to make room for tall ``Submitted Journal Publications''
% marginpar. If there are enough submitted journal publications, this
% space will not be needed (and should be removed).
%\vspace{0.1in}

%\section{Papers in Preparation}
%\vspace{-.1in}
%\begin{bibsection}
%    \item Toomey, T.L., Erickson, D.J., Carlin, B.P., Lenk, K.M., {\bf Quick, H.S.}, and Harwood, E.M. ``Do neighborhood attributes moderate the relationship between alcohol establishment density and crime?"
%    \item {\bf Quick, H.}, Banerjee, S., and Carlin, B.P. ``Heteroscedastic variances in areally referenced temporal processes with an application to California asthma hospitalization data.''

%    \item {\bf Quick, H.}, Carlin, B.P., and Banerjee, S. ``Space-time Gaussian process modeling of temporal air pollution gradients."
%\end{bibsection}

\section{Awards}

%Research Grants
%\begin{innerlist}
%	\item SURFSara Cartesius Computing Grant (author) \hfill Sept 2015- Sept 2016 \\
%	20 million CPU hours awarded (Simon Portegies Zwart PI) \hfill
	
%	\item NASA SMD High End Computing Grant (author) \hfill Sept 2015- Sept 2016 \\
%	4 million CPU hours awarded (Stephen McMillan PI) \hfill
%\end{innerlist}

%\halfblankline

%Graduate Academic Awards
%\begin{innerlist}
%\item Dean's Fellowship for Academic Excellence \hfill Fall 2013, 2 years \\
%	  Graduate College, Drexel University
%\end{innerlist}

%\halfblankline

%Travel Awards
%\begin{innerlist}
%\item International Travel Award for Research, Drexel University\hfill December 2015\\
%MODEST 2015-S, Kobe, Japan

%\item International Travel Award for Research, Drexel University\hfill March 2015\\
%MODEST 2015, Concepci\'{o}n, Chile
%\end{innerlist}


%\halfblankline
\iflong
\textbf{Graduate Academic Awards}\\
Drexel University - Philadelphia
\begin{innerlist}
\item Dept. of Physics Teaching Excellence Award \hfill 2019\\
In recognition of exemplary commitment to student learning.
\end{innerlist}
\fi
\halfblankline

\iflong
\textbf{Undergraduate Academic Awards}\\
California Polytechnic State University - San Luis Obispo
\begin{innerlist}
\item Dept. of Physics Certificate of Excellence: Outstanding Citizenship \hfill May 2016\\
Awarded by Dept. Head Robert S. Echols
\item Dean's List \hfill 7 Quarters\\
Awarded by Dean Philip S. Bailey
\end{innerlist}
\fi


\section{Organizations and \\ Involvement}
%\begin{outerlist}
%\item[] Member and Treasurer of Sigma Pi Sigma: Physics Honors Society\\
%		California Polytechnic State University San Luis Obispo Chapter \hfill 2015 - 2016
%	\begin{innerlist}
%	\item Students must exhibit outstanding academic achievement and be in the top third of their class to be elected as a member. Duties as Treasurer included managing and allocating club funds and coordinating the end of the year banquet with the Madonna Inn.
%	\end{innerlist}
%\item[] American Physical Society (APS), American Astronomical Society (AAS), and \\ Society of Physics Students (SPS)
%\end{outerlist}
\iflong
\textbf{Graduate Leadership}
\begin{innerlist}
\item[] Treasurer and Event Coordinator: \\ Drexel Physics Graduate Student Association (PGSA) \hfill 2019-2020
\end{innerlist}
\fi
\halfblankline

\iflong
\textbf{Undergraduate Leadership}
\begin{innerlist}
\item[] Treasurer and member of Sigma Pi Sigma: Physics Honors Society \hfill 2015 - 2016
%		California Polytechnic State University San Luis Obispo Chapter \hfill 2015 - 2016
\end{innerlist}
\fi
\halfblankline

\iflong
\textbf{Memberships}
\begin{innerlist}
\item[] American Physical Society (APS), American Astronomical Society (AAS), and \\ Society of Physics Students (SPS)
\end{innerlist}
\fi


%\section{Skills}
%Computer Programming:
%\begin{innerlist}
%    \item C$++$, Fortran, Python, LaTex, R, UNIX shell scripting, 
%    GNU make
%\end{innerlist}

%\halfblankline


\section{References}
Stephen McMillan
\begin{innerlist}
\item[] Professor of Physics and Chair \hfill {Phone: 215-895-2709}\\
Department of Physics \hfill{E-mail: steve@physics.drexel.edu }\\
Drexel University
\end{innerlist}

\halfblankline

Brigita Urbanc
\begin{innerlist}
\item[] Academic Advisor \hfill {Phone: 215-895-2726}\\
Department of Physics \hfill {E-mail: bu25@drexel.edu}\\
Drexel University
\end{innerlist}

\halfblankline

Vardha Bennert
\begin{innerlist}
\item[] Associate Professor of Physics \\%\hfill {Phone: 805-756-7317}\\
Department of Physics \hfill {E-mail: vbennert@calpoly.edu}\\
Cal Poly: San Luis Obispo
\end{innerlist}

%Mordecai-Mark Mac Low
%\begin{innerlist}
%\item[] Professor and Curator \hfill {Phone: 212-496-3443}\\
%Department of Astrophysics \hfill{E-mail: mordecai@amnh.org}\\
%American Museum of Natural History
%\end{innerlist}

%\halfblankline

%Alex King
%\begin{innerlist}
%\item[] Professor and Chair \hfill {Phone: 931-221-6102}\\
%Department of Physics and Astronomy \hfill{E-mail: kinga@apsu.edu}\\
%Austin Peay State University
%\end{innerlist}


%\halfblankline

%Justin Oelgoetz
%\begin{innerlist}
%\item[] Professor \hfill {Phone: 931-221-6158}\\
%Department of Physics and Astronomy \hfill{E-mail: oelgoetzj@apsu.edu}\\
%Austin Peay State University
%\end{innerlist}


\end{document}
