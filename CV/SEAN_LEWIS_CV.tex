%----------------------------------------------------------------------------------------
%       Sean C. Lewis
%       Curriculum Vitae
%       Last updated: 29 Nov 2022
%----------------------------------------------------------------------------------------
%	PACKAGES AND OTHER DOCUMENT CONFIGURATIONS
%----------------------------------------------------------------------------------------

\documentclass{resume} % Use the custom resume.cls style

\usepackage[left=0.75in,top=0.6in,right=0.75in,bottom=0.6in]{geometry} % Document margins
\usepackage{hyperref}
\usepackage[usenames,dvipsnames]{xcolor}
\hypersetup{
    colorlinks = true,
    citecolor = {MidnightBlue},
    linkcolor = {BrickRed},
    urlcolor = {BrickRed}
}
\usepackage{enumitem}
\usepackage{etaremune}

%\usepackage{bibentry}

\newcommand{\forceindent}{\leavevmode{\parindent=1em\indent}}

\name{Sean C. Lewis} % Name


\address{Disque Hall, Office No. 808 \\ 32 S. 32$^{\rm nd}$ St. \\ Philadelphia, PA 19104, USA} % Address
\address{+1~$\cdot$~(408)~$\cdot$~470~$\cdot$~0668 \\ \href{mailto:scl63@drexel.edu}{sean.christian.lewis@drexel.edu}} % Phone number and email
\address{Ph.D. Candidate \\ Deptartment of Physics \\ Drexel University } % Title

\pagenumbering{roman}

\begin{document}

  %\nobibliography{CV.bib}
  %\bibliographystyle{unsrt}

%----------------------------------------------------------------------------------------
%	RESEARCH INTERESTS SECTION
%----------------------------------------------------------------------------------------

\begin{rSection}{Research Interests}

\textbf{Computational astrophysics}, including general relativity, gravitational lensing, modified gravity, large-scale structure, 21\,cm cosmology, dark energy, inflation, dark matter, radio astronomy, and gravitational waves. 

\end{rSection}

%----------------------------------------------------------------------------------------
%	EDUCATION SECTION
%----------------------------------------------------------------------------------------

\begin{rSection}{Education}

\textbf{Drexel University} \\%\hfill {2018 -- Present} \\ 
{\color{MidnightBlue} Ph.D.} Student/Candidate of Physics \hfill {2017 -- Present} \\
{\color{MidnightBlue} M.S.} in Physics \hfill{2019}\\
%\textit{GPA: 3.95}
\textbf{California Polytechnic State University} \\%\hfill {2013 -- 2017} \\
{\color{MidnightBlue} B.S.} in Physics \hfill{2016}\\
\textit{Cum Laude}
%\textit{GPA: 3.55 (Cum Laude)}\\
%\textit{Departmental Honors in Physics}

\end{rSection}

%----------------------------------------------------------------------------------------
%	POSITIONS HELD SECTION
%----------------------------------------------------------------------------------------

\begin{rSection}{Positions Held}

\textbf{Drexel University} \hfill {2017 -- Present} \\
\textit{Doctoral Teaching Fellow; Research Fellow}\\
Department of Physics

\textbf{California Polytechnic State University} \hfill {2015 -- 2016}\\
\textit{Research Assistant}\\
Department of Physics

\end{rSection}

%----------------------------------------------------------------------------------------
%	AWARDS AND HONORS SECTION
%----------------------------------------------------------------------------------------

\begin{rSection}{Awards and Honors}

\textit{Chambliss Astronomy Achievement Honorable Mention}, American Astronomical Society \hfill{2020} \\
\textit{Department of Physics Teaching Excellence Award}, Drexel University \hfill{2019} \\
\textit{CoAS Dean Honors List}, California Polytechnic State University \hfill {2012--2016}

\end{rSection}

%----------------------------------------------------------------------------------------
%	RESEARCH HISTORY SECTION
%----------------------------------------------------------------------------------------

\begin{rSection}{Research History}

\begin{description}[leftmargin=8em, style=nextline]

\item[\textnormal{2021-Present}] \textbf{Hydrodynamical Simulation Data Structure Conversion}\\
  Developed a novel software technique for transferring simulation data from a Voronoi Mesh data structure to a block-based adaptively refined grid structure.

\item[\textnormal{2018-Present}] \textbf{Early Forming Massive Stars}\\
  Developed a controlled experiment using the the high performance coupled magnetoydrodynamic, radiation, and N-body software suite Torch to determine the effects of the formation time of very massive stars, an under-tested parameter space.
  Time series data analysis and cluster identification techniques revealed that early forming massive stars had significant effect on star cluster development and evolution.

\end{description}

\end{rSection}

%----------------------------------------------------------------------------------------
%	REFEREED PUBLICATIONS SECTION
%----------------------------------------------------------------------------------------

\begin{rSection}{Refereed Publications}

\begin{etaremune}
\item \textbf{{Lewis}, S.~C.}, {McMillan}, S. L. W., {Mac Low}, M-M., {Cournoyer-Cloutier}, C., {Polak}, B., {Wilhelm}, M. J. C., {Tran}, A., {Sills}, A., {Portegies Zwart}, S., {Klessen} R., and {Wall}, J. E., \textit{``Early Forming Massive Stars Suppress Star Formation and Hierarchical Cluster Assembly,"} Submitted to ApJ (2022) %\href{https://arxiv.org/abs/2207.07784}{[arXiv:2207.07784]}
 
\item {Cournoyer-Cloutier}, C., {Tran}, A., \textbf{{Lewis}, S.~C.}, {Wall}, J.~E., {Harris}, W. E., {Mac Low}, M-M., {McMillan}, S. L. W., {Portegies Zwart}, S., and {Sills}, A., \textit{``Implementing primordial binaries in simulations of star cluster formation with a hybrid MHD and direct N-body method",} MNRAS \textbf{501}, 4464--4478 (2021) \href{https://arxiv.org/abs/2011.06105}{[arXiv:2011.06105]}
  
\item {Bennert}, V., N., {Loveland}, D., {Donohue}, E., {Cosens}, M., \textbf{{Lewis}, S. C.}, {Komossa}, S., {Treu}, T., {Malkan}, M. A., {Milgram}, N., and {Flatland}, K., \textit{``Studying the O III $\lambda$5007 Å emission-line width in a sample of $\sim$ 80 local active galaxies: a surrogate for $\sigma$",} MNRAS. \textbf{481}, 138--152 (2018) \href{https://arxiv.org/abs/1808.04821}{[arXiv:1808.04821]}
\end{etaremune}

\end{rSection}

%----------------------------------------------------------------------------------------
%	CONFERENCE PROCEEDINGS, SCIENCE BOOKS, WHITE PAPERS SECTION
%----------------------------------------------------------------------------------------

%\begin{rSection}{Conference Proceedings, Science Books, White Papers}

%\begin{etaremune}

%\item {Ahmed}, Z., {Alonso}, D., {Amin}, M.~A.,
%         {Ansari}, R., \textbf{{Arena}, E.~J.}, {Bandura}, K.,
%         {Beardsley}, A., {Bull}, P., {Castorina}, E.,
%         {Chang}, T.-C., {Dav{\'e}}, R., {Dillon}, J.~S.,
%         {van Engelen}, A., {Ewall-Wice}, A., {Ferraro}, S.,
%         {Foreman}, S., {Frisch}, J., {Green}, D.,
%         {Holder}, G., {Jacobs}, D., {Karagiannis}, D.,
%         {Kaurov}, A.~A., {Knox}, L., {Kuhn}, E.,
%         {Liu}, A., {Ma}, Y.-Z., {Masui}, K.~W.,
%         {McClintock}, T., {Moodley}, K.,
%         {M{\"u}nchmeyer}, M., {Newburgh}, L.~B.,
%         {Nomerotski}, A., {O'Connor}, P., {Obuljen}, A.,
%         {Padmanabhan}, H., {Parkinson}, D., {Perdereau}, O.,
%         {Rapetti}, D., {Saliwanchik}, B., {Sehgal}, N.,
%         {Shaw}, J.~R., {Sheehy}, C., {Sheldon}, E.,
%         {Shirley}, R., {Silverstein}, E., {Slatyer}, T.,
%         {Slosar}, A., {Stankus}, P., {Stebbins}, A.,
%         {Timbie}, P., {Tucker}, G.~S., {Tyndall}, W.,
%         {Villaescusa-Navarro}, F., and {Wulf}, D.,
%         \textit{``Research and Development for HI Intensity Mapping,''} ArXiv e-prints (2019) \href{https://arxiv.org/abs/1907.13090}{[arXiv:1907.13090]}

%\item {Ahmed}, Z., {Alonso}, D., {Amin}, M.~A.,
%         {Ansari}, R., \textbf{{Arena}, E.~J.}, {Bandura}, K.,
%         {Battaglia}, N, {Blazek}, J., 
%         {Bull}, P., {Castorina}, E.,
%         {Chang}, T.-C., {Connor}, L., {Dav{\'e}}, R., {Dillon}, J.~S.,
%         {Dvorkin}, C.,
%         {van Engelen}, A., {Ferraro}, S., {Flauger}, R.,
%         {Foreman}, S., {Frisch}, J., {Green}, D.,
%         {Holder}, G., {Jacobs}, D., {Johnson}, M.~C.,
%         {Karagiannis}, D.,
%         {Kaurov}, A.~A., {Knox}, L., 
%         {Liu}, A., {Loverde}, M., {Ma}, Y.-Z., {Masui}, K.~W.,
%         {McClintock}, T., {Meerburg}, P.~D., {Moodley}, K.,
%         {M{\"u}nchmeyer}, M., {Newburgh}, L.~B., {Ng}, C.,
%         {Nomerotski}, A., {O'Connor}, P., {Obuljen}, A.,
%         {Padmanabhan}, H., {Parkinson}, D., {Prochaska}, J.~X.,
%         {Rajendran}, S.,
%         {Rapetti}, D., {Saliwanchik}, B., {Schaan}, E., {Sehgal}, N.,
%         {Shaw}, J.~R., {Sheehy}, C., {Sheldon}, E.,
%         {Shirley}, R., {Silverstein}, E., {Slatyer}, T.,
%         {Slosar}, A., {Stankus}, P., {Stebbins}, A.,
%         {Timbie}, P., {Tucker}, G.~S., {Tyndall}, W.,
%         {Villaescusa-Navarro}, F., {Wallisch}, B., and {White}, M.,
%\textit{``Packed Ultra-wideband Mapping Array (PUMA): A Radio Telescope for Cosmology and Transients,''}, Bull.Am.Astron.Soc. \textbf{51}, 53 (2019)  \href{https://arxiv.org/abs/1907.12559}{[arXiv:1907.12559]}


%\item {Cosmic Visions 21$\,$cm Collaboration}, {Ansari}, R., \textbf{{Arena}, E.~J.} , 
%	{Bandura}, K., {Bull}, P., {Castorina}, E., {Chang}, T.-C., 
%	{Foreman}, S., {Frisch}, J., {Green}, D., {Karagiannis}, D., 
%	{Liu}, A., {Masui}, K.~W., {Meerburg}, P.~D., {Newburgh}, L.~B., 
%	{Obuljen}, A., {O'Connor}, P., {Shaw}, J.~R., {Sheehy}, C., 
%	{Slosar}, A., {Smith}, K., {Stankus}, P., {Stebbins}, A., 
%	{Timbie}, P., {Villaescusa-Navarro}, F., and {White}, M., 
%\textit{``Inflation and Early Dark Energy with a {Stage~{\sc ii}} Hydrogen Intensity Mapping experiment,''} ArXiv e-prints (2018) \href{https://arxiv.org/abs/1810.09572}{[arXiv:1810.09572]}

%\end{etaremune}

%\end{rSection}

%----------------------------------------------------------------------------------------
%	TALKS AND PRESENTATIONS SECTION
%----------------------------------------------------------------------------------------
%\newpage %Just for now until more pubs. are added

\begin{rSection}{Conferences and Talks}

\textbf{Contributed Talks}\\
``Hybrid analytic image modeling and image moments approach to gravitational lensing''\\
\forceindent Public talk for my Phyics Ph.D. Candidacy Exam, Drexel University \hfill 4 Jun. 2020\\
``Quantifying the Effects of O-type Star Formation in Embedded Stellar Clusters''\\
\forceindent Modest 21a Virtual Conference \hfill Jul. 2021\\

\textbf{Poster Presentations}\\
- ``The Effects of Early Massive Star Formation: Gas Expulsion and Cluster Dynamics''\\
\forceindent American Astronomical Society -- 238th Conference \hfill Jun. 2021\\
- ``The effects of O-type star formation in embedded stellar clusters.''\\
\forceindent American Astronomical Society -- 236th Conference \hfill Jun. 2020\\
- ``Was the first observed hypervelocity globular cluster, \\ \forceindent HVGC-1, accelerated by a supermassive binary black hole?"\\
\forceindent American Astronomical Society -- 233rd Conference \hfill Jan. 2019\\
- ``The mystery of a hypervelocity globular cluster: is a binary black hole to blame?"\\
\forceindent Drexel Emerging Graduate Scholars, Drexel University \hfill Sept. 2018\\

\end{rSection}

%----------------------------------------------------------------------------------------
%	SOFTWARE SECTION
%----------------------------------------------------------------------------------------

\begin{rSection}{Software Developed}

\underline{\makebox[0.965\textwidth][l]{\textbf{Authored}}}

\begin{description}[leftmargin=10em, style=nextline]

\item[\texttt{VorAMR}] A robust tool that utilizes \texttt{scipy} nearest neighbor interpolation and the \href{https://www.amusecode.org/}{AMUSE} software suite to convert output data from any Voronoi mesh data structure to input data for adaptive block-based structures.  \textit{Publicly available code written in Python}. \href{https://bitbucket.org/torch-sf/vor-amr/src/main/}{https://bitbucket.org/torch-sf/voramr/src/main}

\item[\texttt{PythonOpenMPI}] A generalizable utility for efficient task-based parallel programming using the \texttt{mpi4py} library. \textit{Publicly available code written in Python}.\\ \href{https://github.com/seanlabean/PythonOpenMPI}{https://github.com/seanlabean/PythonOpenMPI}

\end{description}

\underline{\makebox[0.965\textwidth][l]{\textbf{Contributed}}}

\begin{description}[leftmargin=10em, style=nextline]

\item[\texttt{Torch}] A star cluster formation simulation software suite that couples the \href{https://www.amusecode.org/}{AMUSE} framework with the magnetohydrodynamical code  \href{https://flash.rochester.edu/site/flashcode/user_support/}{FLASH}.  \textit{Publicly available code written in Python}. \href{https://github.com/apetri/LensTools}{https://github.com/apetri/LensTools}

\end{description}

\end{rSection}

%----------------------------------------------------------------------------------------
%	TEACHING SECTION
%----------------------------------------------------------------------------------------

\begin{rSection}{Teaching}

\textbf{Drexel University} \\ 
\textit{Teaching Assistant} (Recitation and Lab Instructor)\\
\forceindent PHYS 100, \textit{Preparation for Engineering Studies}  \hfill {Winter: 2021, 2020, 2019}\\
\forceindent PHYS 152, \textit{Introductory Physics I}  \hfill {Spring: 2022, 2021, 2020, 2019}\\
\forceindent PHYS 154, \textit{Introductory Physics III}  \hfill {Fall: 2021, 2020, 2019, 2018}\\
\textit{Grader} \\
\forceindent PHYS 131, \textit{Survey of the Universe} \hfill {Winter 2022}\\
\forceindent PHYS 231, \textit{Introductory Astrophysics} \hfill {Winter 2022}\\
\textit{Guest Lecturer} \\
\forceindent PHYS 231, \textit{Introductory Astrophysics} \hfill {Winter 2022}\\
\textbf{Stony Brook University} \\
\textit{Lecturer}\\
\forceindent  Della Pietra High School Applied Math Program \hfill {Spring 2017}

\end{rSection}

%----------------------------------------------------------------------------------------
%	PROFESSIONAL ACTIVITIES AND SERVICE SECTION
%----------------------------------------------------------------------------------------

%\newpage %Just for now

\begin{rSection}{Professional Activities and Service}

\begin{description}[leftmargin=10em, style=nextline]

\item[Collaborations] External Collaborator, Dark Energy Survey (DES)\\
  Member, Packed Ultra-wideband Mapping Array (PUMA) [Inactive]\\
  Member, Baryon Mapping eXperiment (BMX) [Inactive]

\item[Working Groups] Member, DOE Cosmic Visions Dark Energy 21$\,$cm Working Group [Inactive]

%\item[Collaborations] Inactive member of the Large Synoptic Survey Telescope Dark Energy 
% Science Collaboration (LSST-DESC)

\end{description}

\textbf{Outreach Activities}\\
Invited to appear on the Drexel University Teaching Assistant Orientation Panel, as part of the Teaching Assistant Orientation and Preparation Course GRAD T580 (17 Sep. 2020).

Gave a physics demonstration at the Kaczmarczik Lecture Series Open House, hosted by the Drexel University Department of Physics (14 Nov. 2018).

\textbf{Committee Work}\\
Treasurer of the Drexel University Physics Graduate Student Association (2020 -- 2021).

\end{rSection}


%----------------------------------------------------------------------------------------
%	TECHNICAL SKILLS SECTION
%----------------------------------------------------------------------------------------

%\begin{rSection}{Technical Skills}
%
%\begin{description}[leftmargin=16em, style=nextline]
%
%\item[Computer Languages] Python, Fortran
%\item[Tools] Mathematica
%\item[Markup] \LaTeX, HTML, CSS

%\end{description}

%\end{rSection}


\end{document}
