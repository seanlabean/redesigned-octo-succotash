%----------------------------------------------------------------------------------------
%       Sean C. Lewis
%       Curriculum Vitae
%       Last updated: 29 Nov 2022
%----------------------------------------------------------------------------------------
%	PACKAGES AND OTHER DOCUMENT CONFIGURATIONS
%----------------------------------------------------------------------------------------

\documentclass{resume} % Use the custom resume.cls style

\usepackage[left=0.75in,top=0.6in,right=0.75in,bottom=0.6in]{geometry} % Document margins
\usepackage{hyperref}
\usepackage[usenames,dvipsnames]{xcolor}
\hypersetup{
    colorlinks = true,
    citecolor = {MidnightBlue},
    linkcolor = {BrickRed},
    urlcolor = {BrickRed}
}
\usepackage{enumitem}
\usepackage{etaremune}

%\usepackage{bibentry}

\newcommand{\forceindent}{\leavevmode{\parindent=1em\indent}}

\name{Sean C. Lewis} % Name


\address{Philadelphia, PA 19104, USA} % Address
\address{\href{mailto:sean.phys@gmail.com}{sean.phys@gmail.com}} % Phone number and email
\address{Ph.D. Candidate \\ Deptartment of Physics \\ Drexel University } % Title

\pagenumbering{roman}

\begin{document}

  %\nobibliography{CV.bib}
  %\bibliographystyle{unsrt}

%----------------------------------------------------------------------------------------
%	RESEARCH INTERESTS SECTION
%----------------------------------------------------------------------------------------

\begin{rSection}{Research Interests}

\textbf{Computational physics}, including the use of Lagrangian grid codes, magnetohydrodynamics, and radiative transfer to model the formation of evolution of star clusters, the impact of massive stars and stellar feedback, and the pursuit of self-consistent star formation. In addition: to employ and enhance Lagrangian models to study high energy events.

\end{rSection}

%----------------------------------------------------------------------------------------
%	EDUCATION SECTION
%----------------------------------------------------------------------------------------

\begin{rSection}{Education}

\textbf{Drexel University} \\%\hfill {2018 -- Present} \\ 
{\color{MidnightBlue} Ph.D.} Student/Candidate of Physics \hfill {2017 -- Present} \\
{\color{MidnightBlue} M.S.} in Physics \hfill{2019}\\
%\textit{GPA: 3.95}
\textbf{California Polytechnic State University} \\%\hfill {2013 -- 2017} \\
{\color{MidnightBlue} B.S.} in Physics \hfill{2016}\\
\textit{Cum Laude}
%\textit{GPA: 3.55 (Cum Laude)}\\
%\textit{Departmental Honors in Physics}

\end{rSection}

%----------------------------------------------------------------------------------------
%	POSITIONS HELD SECTION
%----------------------------------------------------------------------------------------

\begin{rSection}{Positions Held}

\textbf{Drexel University} \hfill {2017 -- Present} \\
\textit{Teaching Fellow; Research Fellow}\\
Department of Physics

\textbf{California Polytechnic State University} \hfill {2015 -- 2016}\\
\textit{Research Assistant}\\
Department of Physics

\end{rSection}

%----------------------------------------------------------------------------------------
%	AWARDS AND HONORS SECTION
%----------------------------------------------------------------------------------------

\begin{rSection}{Awards and Honors}

\textit{Chambliss Astronomy Achievement Honorable Mention}, American Astronomical Society \hfill{2020} \\
\textit{Department of Physics Teaching Excellence Award}, Drexel University \hfill{2019} \\
\textit{CoAS Dean Honors List}, California Polytechnic State University \hfill {2012--2016}

\end{rSection}

%----------------------------------------------------------------------------------------
%	RESEARCH HISTORY SECTION
%----------------------------------------------------------------------------------------

\begin{rSection}{Research History}

\begin{description}[leftmargin=8em, style=nextline]

\item[\textnormal{2021-Present}] \textbf{Hydrodynamical Simulation Data Structure Conversion}\\
  Developed a novel software technique for transferring simulation data from a Voronoi mesh data structure to a block-based adaptively refined grid structure utilizing matrix manipulation and interpolation.

\item[\textnormal{2018-Present}] \textbf{Early Forming Massive Stars}\\
  Developed a controlled experiment using the the high performance coupled magnetoydrodynamic, radiation, and N-body software suite Torch to determine the effects of the formation time of very massive stars, an under-tested parameter space.
  %Time series data analysis and cluster identification techniques revealed that early forming massive stars had significant effect on star cluster development and evolution.

\end{description}

\end{rSection}

%----------------------------------------------------------------------------------------
%	REFEREED PUBLICATIONS SECTION
%----------------------------------------------------------------------------------------

\begin{rSection}{Refereed Publications}

\begin{etaremune}
\item {Cournoyer-Cloutier}, C., {Sills}, A., {Harris}, W.~E., {Appel}, S., \textbf{{Lewis}, S.~C.}, {Polak}, B., {Wilhelm}, M.~J.~C., {Mac Low}, M-M., {McMillan}, S.~L.~W., {Portegies Zwart}, S., \textit{``Early Evolution and 3D structure of Embedded Star Clusters,''} MNRAS \textbf{521}, 1338--1352 (2023)

\item {Wilhelm}, M.~J.~C., {Portegies Zwart}, S., {Cournoyer-Cloutier}, C., \textbf{{Lewis}, S.~C.}, {Polak}, B., {Tran}, A., {Mac Low}, M-M., {McMillan}, S.~L.~W., \textit{``Radiation shielding of protoplanetary discs in your star-forming regions'',} MNRAS \textbf{520}, 5331--5353 (2023)
 
 \item \textbf{{Lewis}, S.~C.}, {McMillan}, S. L. W., {Mac Low}, M-M., {Cournoyer-Cloutier}, C., {Polak}, B., {Wilhelm}, M.~J.~C., {Tran}, A., {Sills}, A., {Portegies Zwart}, S., {Klessen} R., and {Wall}, J.~E., \textit{``Early Forming Massive Stars Suppress Star Formation and Hierarchical Cluster Assembly",} ApJ \textbf{944}, 211 (2023) %\href{https://arxiv.org/abs/2212.01465}{[arXiv:2212.01465]}
 
 \item {Wilhelm}, M.~J.~C., {Portegies Zwart}, S., {Cournoyer-Cloutier}, C., \textbf{{Lewis}, S.~C.}, {Polak}, B., {Tran}, A., {Mac Low}, M-M., and {McMillan}, S.~L.~W., \textit{``Modeling protoplanetary disk evolution in young star forming regions",} Proceedings of the International Astronomical Union, \textbf{16(S362)}, 300--305 (2023)
 
\item {Cournoyer-Cloutier}, C., {Tran}, A., \textbf{{Lewis}, S.~C.}, {Wall}, J.~E., {Harris}, W. E., {Mac Low}, M-M., {McMillan}, S.~L.~W., {Portegies Zwart}, S., and {Sills}, A., \textit{``Implementing primordial binaries in simulations of star cluster formation with a hybrid MHD and direct N-body method",} MNRAS \textbf{501}, 4464--4478 (2021) %\href{https://arxiv.org/abs/2011.06105}{[arXiv:2011.06105]}
  
\item {Bennert}, V., N., {Loveland}, D., {Donohue}, E., {Cosens}, M., \textbf{{Lewis}, S.~C.}, {Komossa}, S., {Treu}, T., {Malkan}, M. A., {Milgram}, N., and {Flatland}, K., \textit{``Studying the O III $\lambda$5007 Å emission-line width in a sample of $\sim$ 80 local active galaxies: a surrogate for $\sigma$",} MNRAS \textbf{481}, 138--152 (2018) %\href{https://arxiv.org/abs/1808.04821}{[arXiv:1808.04821]}
\end{etaremune}

\end{rSection}

%----------------------------------------------------------------------------------------
%	REFEREED PUBLICATIONS SECTION
%----------------------------------------------------------------------------------------

\begin{rSection}{Papers in prep}
\begin{etaremune}

\item \textbf{{Lewis}, S.~C.}, {Mac Low}, M-M., {McMillan}, S. L. W., Li, H., \textit{``VorAMR: transfer of data from a Voronoi mesh to an adaptive mesh for self-consistent top-down star formation,"} to be submitted (2023)

\item \textbf{{Lewis}, S.~C.}, {Mac Low}, M-M., {McMillan}, S. L. W., Li, H., \textit{``Star-by-star cluster formation from self-consistent giant molecular clouds,"} to be submitted (2023)

\end{etaremune}
\end{rSection}

\begin{rSection}{Co-PI Grants}
\begin{etaremune}
\item Accelerate ACCESS PHY220160: \textit{``Models of Star Cluster Formation Using a Multiphysics Framework"} - 1.675 Million credits awarded \hfill Jan.2023
\end{etaremune}
\end{rSection}

%----------------------------------------------------------------------------------------
%	TALKS AND PRESENTATIONS SECTION
%----------------------------------------------------------------------------------------
%\newpage %Just for now until more pubs. are added

\begin{rSection}{Conferences and Talks}

\underline{\makebox[0.965\textwidth][l]{\textbf{Contributed Talks}}}\\
--``Star Cluster Formation: The effects of early forming massive stars and building \\ \forceindent a bridge between Voronoi mesh and block-structured codes"\\
\forceindent American Astronomical Society -- 241st Conference \hfill Jan. 2023\\
--``The Effects of Early Forming Massive Stars \& A Novel Method for \\ \forceindent Inter-codebase Interpolation''\\
\forceindent Clusters 2022, McMaster University \hfill 23 Aug. 2022\\
-- ``Quantifying the Effects of O-type Star Formation in Embedded Stellar Clusters''\\
\forceindent Modest 21a Virtual Conference \hfill Jul. 2021\\
--``Using the MHD code FLASH to create a protoplanetary disk''\\
\forceindent Phyics Ph.D. Candidacy Exam, Drexel University \hfill 4 Jun. 2019\\

\underline{\makebox[0.965\textwidth][l]{\textbf{Poster Presentations}}}\\
-- ``The Effects of Early Massive Star Formation: Gas Expulsion and Cluster Dynamics''\\
\forceindent American Astronomical Society -- 238th Conference \hfill Jun. 2021\\
-- ``The effects of O-type star formation in embedded stellar clusters.''\\
\forceindent American Astronomical Society -- 236th Conference \hfill Jun. 2020\\
-- ``Was the first observed hypervelocity globular cluster, \\ \forceindent HVGC-1, accelerated by a supermassive binary black hole?"\\
\forceindent American Astronomical Society -- 233rd Conference \hfill Jan. 2019
%- ``The mystery of a hypervelocity globular cluster: is a binary black hole to blame?"\\
%\forceindent Drexel Emerging Graduate Scholars, Drexel University \hfill Sept. 2018\\

\end{rSection}

%----------------------------------------------------------------------------------------
%	SOFTWARE SECTION
%----------------------------------------------------------------------------------------

\begin{rSection}{Software Developed}

\underline{\makebox[0.965\textwidth][l]{\textbf{Authored}}}

\begin{description}[leftmargin=10em, style=nextline]

\item[\texttt{VorAMR}] A robust tool that utilizes \texttt{numpy} matrix manipulation, \texttt{scipy} nearest neighbor interpolation and the \href{https://www.amusecode.org/}{AMUSE} software suite to convert output data from any Voronoi mesh data structure to input data for adaptive block-based structures.  \textit{Publicly available code written in Python}. \href{https://bitbucket.org/torch-sf/torch/src/vorch/src/voramr/}{[Bitbucket Repository]}

\item[\texttt{PythonOpenMPI}] A generalizable utility for efficient task-based parallel programming using the \texttt{mpi4py} library. \textit{Publicly available code written in Python}.\\ \href{https://github.com/seanlabean/PythonOpenMPI}{[Github Repository]}

\end{description}

\underline{\makebox[0.965\textwidth][l]{\textbf{Contributed}}}

\begin{description}[leftmargin=10em, style=nextline]

\item[\texttt{Torch}] A star cluster formation simulation software suite that couples the \href{https://www.amusecode.org/}{AMUSE} framework with the magnetohydrodynamical code  \href{https://flash.rochester.edu/site/flashcode/user_support/}{FLASH}.  \textit{Publicly available code written in Python}. \href{https://bitbucket.org/torch-sf/torch/src/main/src/}{[Bitbucket Repository]}

\end{description}

\end{rSection}

%----------------------------------------------------------------------------------------
%	TEACHING SECTION
%----------------------------------------------------------------------------------------

%\begin{rSection}{Teaching}

%\textbf{Drexel University} \\ 
%\textit{Teaching Fellow} (Recitation)\\
%\forceindent PHYS 121, \textit{Physics for Art \& Design Students}\\  %\hfill {Winter: 2021, 2020, 2019}\\
%\forceindent PHYS 101, \textit{Introduction to Calculus-based Physics I}\\  %\hfill {Spring: 2022, 2021, 2020, 2019}\\
%\forceindent PHYS 102, \textit{Introduction to Calculus-based Physics II}\\
%\forceindent PHYS 201, \textit{Introduction to Calculus-based Physics III}\\  %\hfill {Fall: 2021, 2020, 2019, 2018}\\
%\textit{Grader} \\
%\forceindent PHYS 131, \textit{Survey of the Universe}\\ %\hfill {Winter 2022}\\
%\forceindent PHYS 231, \textit{Introductory Astrophysics}\\ %\hfill {Winter 2022}\\

%\end{rSection}

%----------------------------------------------------------------------------------------
%	PROFESSIONAL ACTIVITIES AND SERVICE SECTION
%----------------------------------------------------------------------------------------

%\newpage %Just for now

%\begin{rSection}{Professional Activities and Service}

%\begin{description}[leftmargin=10em, style=nextline]

%\item[Collaborations] External Collaborator, Dark Energy Survey (DES)\\
%  Member, Packed Ultra-wideband Mapping Array (PUMA) [Inactive]\\
%  Member, Baryon Mapping eXperiment (BMX) [Inactive]

%\item[Working Groups] Member, DOE Cosmic Visions Dark Energy 21$\,$cm Working Group [Inactive]

%\item[Collaborations] Inactive member of the Large Synoptic Survey Telescope Dark Energy 
% Science Collaboration (LSST-DESC)

%\end{description}

%\textbf{Outreach Activities}\\
%Invited to appear on the Drexel University Teaching Assistant Orientation Panel, as part of the Teaching Assistant Orientation and Preparation Course GRAD T580 (17 Sep. 2020).

%Gave a physics demonstration at the Kaczmarczik Lecture Series Open House, hosted by the Drexel University Department of Physics (14 Nov. 2018).

%\textbf{Committee Work}\\
%Treasurer \& Event Coordinator of the Drexel Physics Graduate Student Association (2019 -- 2020).

%\end{rSection}


%----------------------------------------------------------------------------------------
%	TECHNICAL SKILLS SECTION
%----------------------------------------------------------------------------------------

%\begin{rSection}{Technical Skills}
%
%\begin{description}[leftmargin=16em, style=nextline]
%
%\item[Computer Languages] Python, Fortran
%\item[Tools] Mathematica
%\item[Markup] \LaTeX, HTML, CSS

%\end{description}

%\end{rSection}


\end{document}
